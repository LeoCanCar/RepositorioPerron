\documentclass[10pt,a4paper]{article}
\usepackage[utf8]{inputenc}
\usepackage{amsmath}
\usepackage{amssymb}

%opening
\title{Tarea 1 en LaTeX}
\author{Por: Leonardo Canseco Carbajal}

\begin{document}
\maketitle 
% \begin{abstract}
% 
% \end{abstract}

\section{Seccion 1}

La verdad no sabía que existía LaTeX hasta la semana antepasada, pero la verdad está muy perron.\\

Primeras ecuaciones en \LaTeX.\\ % Cambia este texto por algun texto tuyo.

Ecuación del profe\\
$ e^{i \pi} + 1 = 0 $ \\ % Comenta esta ecacución y escribe abajo una ecuación muy simple (la que quieras).

Mi ecuación\\
Ley de Planck:\\
$$I(v,T)= \dfrac{2hv^3}{c^2} \dfrac{1}{e^{\frac{hv}{kT}} -1}$$
Donde:\\

\textit{I,I' (Radiancia espectral)}, es la cantidad de energía por unidad de superficie, unidad de tiempo y unidad de ángulo sólido por unidad de frecuencia o longutid de onda\\
\textit{v}, frecuencia\\
\textit{T}, temperatura del cuerpo negro\\
\textit{h}, constante de Planck\\
\textit{c}, velocidad de la luz\\
\textit{e}, base del logaritmo natural\\
\textit{k}, constante de Boltzmann\\

\end{document}